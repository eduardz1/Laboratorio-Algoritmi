\documentclass[12pt, letterpaper]{report}
\usepackage[utf8]{inputenc}
\usepackage{tikz}

\title{Relazione Algoritmi e Strutture Dati}
\author{Eduard Antonovic Occhipinti, Iman Solaih, Marco Molica}

\begin{document}
\maketitle
\tableofcontents

\chapter*{Esercizio 1}
\section{Quick Sort}
Il \verb|quick_sort()| è un algoritmo che ordina una collezione partendo da un pivot, 
il pivot può essere scelto in vari modi, e in base a quale viene scelto il tempo
di sorting varia. Il \verb|quick_sort()| utilizza \verb|_part()| per scegliere il pivot prima 
di chiamare \verb|partition()| per dividere gli elementi del range selezionato 
in un sottoinsieme di elementi maggiori e uno di elementi minori del pivot
la cui posizione finale viene restituita dal metodo.

\subsection{Impatto della scelta del pivot nel quick sort}
La chiamata a \verb|rand()| porta il \verb|quick_sort()| con pivot scelto 
randomicamente o come mediana di tre numeri ad essere mediamente più lento 
rispetto agli altri 3 casi presi in considerazione. La tabella sottostante 
riporta il tempo impiegato ad ordinare un array di 20 milioni elementi di tipo \verb|struct Record|

\chapter*{Esercizio 2}
\section{Binary Insertion Sort}
\section{Skip List}
\section{Minimum Heap}
\section{Graph}

\end{document}